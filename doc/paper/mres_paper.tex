
\documentclass{amsart}

\usepackage{graphicx}
%\usepackage[altbullet]{lucidabr}
%two lines below change font (font intalled manually (i.e. uploaded))
%\usepackage{fontspec}
%\setmainfont[Ligatures=TeX]{LucidaBrightRegular.ttf}
%\usepackage{kpfonts}    % for nice fonts
% option [light] for more aery documents
\usepackage{color}  %for color of references
\usepackage[dvipsnames]{xcolor} %for color of references
\usepackage{caption}
\usepackage{fancyhdr}
\usepackage[pagebackref,colorlinks, citecolor=Brown,urlcolor=Brown]{hyperref}
\usepackage{natbib}
\usepackage{multicol}
\usepackage{multirow}
%\usepackage{lscape}
\usepackage{pdflscape}
\usepackage{amssymb}
\usepackage{geometry}
\usepackage{longtable}
\usepackage{colortbl}
\usepackage{dsfont}
\usepackage{bm}
\usepackage{mathtools}
\usepackage{pgf}
\usepackage{tikz}
\usepackage{soul}
\usepackage{tikz}
\usepackage{tikz,fullpage}
\usepackage{pgf}
\usepackage{tikz}
\usetikzlibrary{shapes.geometric, arrows} %to create flow charts
\usepackage{bold-extra} %for bold small caps in the title
\usepackage{dirtree} % to create lists as tree

%\renewcommand{\familydefault}{\sfdefault} %for the sans serif font

%AMS original setup for mathematical elements
\newtheorem{theorem}{Theorem}[section]
\newtheorem{lemma}[theorem]{Lemma}
\theoremstyle{definition}
\newtheorem{definition}[theorem]{Definition}
\newtheorem{example}[theorem]{Example}
\newtheorem{xca}[theorem]{Exercise}
\theoremstyle{remark}
\newtheorem{remark}[theorem]{Remark}
\numberwithin{equation}{section}

%    Absolute value notation
\newcommand{\abs}[1]{\lvert#1\rvert}

%    Blank box placeholder for figures (to avoid requiring any
%    particular graphics capabilities for printing this document).
\newcommand{\blankbox}[2]{%
  \parbox{\columnwidth}{\centering
%    Set fboxsep to 0 so that the actual size of the box will match the
%    given measurements more closely.
    \setlength{\fboxsep}{0pt}%
    \fbox{\raisebox{0pt}[#2]{\hspace{#1}}}%
  }%
}

%Tikz setup for a flow chart
\tikzstyle{modelblock} = [rectangle, rounded corners, minimum width=3cm, minimum height=1cm,text centered, draw=black, fill=white, text ragged]

\tikzstyle{arrow} = [thick,->,>=stealth]

\begin{document}

\title{MRes Paper Draft}

%    Information for first author
\author{Arnaud Dy\`evre}
%\address{}
%\curraddr{}
%\email{a.dyevre@lse.ac.uk}
%\thanks{}

%    Information for second author
%\author{}
%\address{}
%\email{}
%\thanks{}

%    General info
%\subjclass[2000]{}

\date{\today. First created October 19, 2019}

%\dedicatory{}
%\keywords{}

%\begin{abstract}

%\end{abstract}

\maketitle

\begin{center}
     Preliminary and incomplete -- Not for sharing
\end{center}


\vspace{12pt}

%% The correct journal style for \specialsection is all uppercase; a known bug
%% in amsart.cls prevents this, so input must be uppercase until it is fixed.
%\specialsection*{This is a Special Section Head}
%\specialsection*{THIS IS A SPECIAL SECTION HEAD}
%This is an example of a special section head%
%%%%%%%%%%%%%%%%%%%%%%%%%%%%%%%%%%%%%%%%%%%%%%%%%%%%%%%%%%%%%%%%%%%%%%%%
%\footnote{Here is an example of a footnote. Notice that this footnote text is running on so that it can stand as an example of how a footnote with separate paragraphs should be written.
%\par
%And here is the beginning of the second paragraph.}%
%%%%%%%%%%%%%%%%%%%%%%%%%%%%%%%%%%%%%%%%%%%%%%%%%%%%%%%%%%%%%%%%%%%%%%%%


\tableofcontents

\newpage

\section{Literature review}

\cite{magerman2016heterogeneous} build a model of heterogeneous firms interlinked in a production network. They calibrate the model using the universe of firm-to-firm transactions in Belgium and conclude that two key aspects of micro-level heterogeneity lead to macro fluctuations: asymmetries in buyer-supplier linkages and share of sales going to final consumption. The relative contribution of each channel depends on the labour share in the economy. Uses monopolistic competition \textit{à la} Melitz (2003) and propagation of shocks on static networks \textit{à la} Acemoglu. There is however no dynamic network formation aspect to this model. \\

\cite{baqaee2017productivity} derive formulas for the spread of inefficiencies in a production network. They note that Hulten's theorem \citep{hulten1978growth} is the benchmark result in the recent literature on production networks. It builds on the fact that firms are cost-minimisers, they can thus leverage micro envelope conditions, instead of macro ones, as Hulten does. Changes in TFP can be decomposed as changes in pure, exogenous changes in technology and endogenous changes in allocative efficiency.

\section{Data}

Statistics on the universe of Belgian firms can be used to compare the \textit{Compustat} coverage \citep{magerman2016heterogeneous}.

\section{Theory}

\cite{magerman2016heterogeneous} use a fixed network for the theoretical and empirical exercises. But using a smaller time span than me, and not interested in the long-term evolution of the network. Also consider that extensive margin does not matter compare to intensive margin when considering spread of shocks. \\

But here, I do not care about the spread of shocks. Mostly concerned with the evolution of the network.

Production network as 

\newpage

\bibliographystyle{ecta}
\bibliography{mres_paper}

\end{document}

%------------------------------------------------------------------------------
% End of journal.tex
%------------------------------------------------------------------------------
